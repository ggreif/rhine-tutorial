\documentclass[handout]{enigtex-beamer-base}

\useoutertheme{miniframes}


\usepackage{comment}
\usepackage[T1]{fontenc}
\usepackage[utf8]{inputenc}

\usepackage{minted}

\title{Rhine - FRP with type level clocks}

\subtitle{A tea tutorial}

\author{Manuel Bärenz}

\begin{document}

\begin{frame}
\titlepage
\end{frame}

\section{Setup}

\begin{frame}[fragile]
\note<.>{Open a console and run these commands to set up the tutorial environment.
While you do so, I'm going to introduce you to Rhine and explain the plan for the tutorial.}
\begin{verbatim}
cabal update
git clone https://github.com/turion/rhine-tutorial/
cd rhine-tutorial
cabal sandbox init
cabal install --only-dependencies
cabal configure
cabal build
cabal run rhine-tutorial
\end{verbatim}
\end{frame}

\section{Quick introduction to Rhine}

\subsection{Synchronous arrowized FRP}

\begin{frame}[fragile]
\begin{block}<+->{\texttt{Dunai} (Iván Pérez, Henrik Nilsson, MB)}
\begin{minted}{haskell}
data MSF m a b = MSF (a -> m (b, MSF m a b))
\end{minted}
\end{block}
\begin{block}<+->{}
\begin{minted}{haskell}
-- Control.Arrow
(>>>) :: MSF m a b -> MSF m b c -> MSF m a c
(***) :: MSF m a b -> MSF m c d -> MSF m (a,c) (b,d)
arr :: (a -> b) -> MSF m a b
\end{minted}
\end{block}
\begin{block}<+->{}
\begin{minted}{haskell}
-- only dunai
arrM :: (a -> m b) -> MSF m a b
\end{minted}
\end{block}
\begin{itemize}[<+->]
	\item \mintinline{haskell}{MSF (Reader Double)} is a replacement for \texttt{Yampa}.
	\item Other monads allow for concise FRP paradigms:
		\begin{itemize}[<+->]
			\item \mintinline{haskell}{State}, \mintinline{haskell}{Reader} and \mintinline{haskell}{Writer} give global state variables.
			\item \mintinline{haskell}{List} gives branching computations.
			\item \mintinline{haskell}{Either} gives control flow!
		\end{itemize}
	\item Support for (entering and leaving) monad transformers.
\end{itemize}
\end{frame}

\begin{frame}[fragile]
\begin{block}<+->{Arrow syntax}
\begin{minted}{haskell}
{-# LANGUAGE Arrows #-}

verboseSum :: MSF IO Int Int
verboseSum = proc n -> do
  s <- sumS       -< n
  _ <- arrM print -< "The sum is now " ++ show s
  returnA         -< s
\end{minted}
\end{block}
\end{frame}

\begin{frame}[fragile]
	% TODO Copy from abstract
	\begin{description}[<+->]
		\item[Clock type] Describes when, and how often,
		the clock should tick.
		\item[Clock value] Implementation details
		\begin{itemize}[<+->]
			\item E.g. physical device address
			\item Implementation choice
		\end{itemize}
		\item[Running clock] Side-effectful stream of \emph{time stamps}, tagged with additional info about the \emph{tick}.
	\end{description}
\end{frame}


\begin{frame}[fragile]
\begin{block}<+->{\texttt{Rhine}}
\begin{minted}{haskell}
-- simplified here
class Clock m cl where
  type TimeDomainOf cl -- time stamp
  type Tag cl -- additional information about tick
  startClock :: cl -> MSF m () (TimeInfo cl, Tag cl)
\end{minted}
\end{block}

\begin{block}<+->{}
\begin{minted}{haskell}
data TimeInfo cl = {...}
  -- absolute and relative time, tag
\end{minted}
\end{block}
\note<.>{While a clock only needs to output a time stamp,
\texttt{Rhine} calculates absolute and relative (to the start of the program and to the last tick) time.}
\end{frame}


\begin{frame}[fragile]
\visible<+->{A clock produces side effects to...}
\begin{itemize}
	\item[...] wait between ticks,
	\item[...] measure the current time,
	\item[...] produce additional data (e.g. events).
\end{itemize}

\begin{block}<+->{Examples of clocks}
\begin{itemize}
	\item Fixed sample rate (e.g. \mintinline{haskell}{Millisecond n})
	\item Events (e.g. \mintinline{haskell}{Stdin})
\end{itemize}
\end{block}
\end{frame}


\begin{frame}[fragile]
\begin{block}<+->{}
\begin{minted}{haskell}
type SyncSF m cl a b = MSF (ReaderT (TimeInfo cl)) a b
\end{minted}
\end{block}
\begin{block}<+->{Lifting \texttt{dunai} concepts}
\begin{minted}{haskell}
arrMSync :: (a -> m b) -> SyncSF m cl a b
\end{minted}
\end{block}
\begin{block}<+->{Time information}
\begin{minted}{haskell}
timeInfo :: SyncSF m cl () (TimeInfo cl)
\end{minted}
\end{block}
\begin{block}<+->{Basic signal processing}
\begin{minted}{haskell}
integral :: VectorSpace v => SyncSF m cl v v
\end{minted}
\end{block}
\end{frame}

\subsection{Exceptions and control flow}

\begin{frame}[fragile]
\begin{block}<+->{\mintinline{haskell}{ExceptT}...}
\begin{minted}{haskell}
--base, transformers
data Either e a = Left e | Right a
newtype ExceptT e m a = ExceptT (m (Either e a))
\end{minted}
\end{block}

\begin{block}<+->{...control flow! (Thanks to Paolo Capriotti)}
\begin{minted}{haskell}
-- dunai, rhine (simplified)
newtype SyncExcept m cl a b e
  = SyncExcept (SyncSF (ExceptT e m) cl a b)
\end{minted}
\end{block}

\begin{block}<+->{}
\begin{minted}{haskell}
instance Monad m => Monad (SyncExcept m cl a b)

throwOn' :: SyncSF (ExceptT e m) cl (Bool, e) ()
try :: SyncSF (ExceptT e m) cl a b
    -> SyncExcept m cl a b e
safely :: SyncExcept m cl a b Empty -> SyncSF m cl a b
safe :: SyncSF m cl a b -> SyncExcept m cl a b e
\end{minted}
\end{block}
\end{frame}

\begin{frame}[fragile]
\begin{block}<+->{Hello World!}
\begin{minted}{haskell}
type SumClock = Millisecond 100

fillUp :: SyncSF (ExceptT Double m) SumClock Double ()
fillUp = proc x -> do
  s <- integral -< x
  _ <- throwOn' -< (s > 5, s)
  returnA       -< ()

helloWorld :: SyncExcept IO SumClock () () Empty
helloWorld = do
  try $ arr (const 1) >>> fillUp
  once_ $ putStrLn "Hello World!"
  helloWorld

main = flow $ safely helloWorld @@ waitClock
\end{minted}
\end{block}
\end{frame}

\begin{frame}[fragile]
\begin{block}{Clock safety}
\begin{minted}{haskell}
fastSignal    :: SyncSF m FastClock () a

slowProcessor :: SyncSF m SlowClock    a b

clockTypeError = fastSignal >>> slowProcessor
\end{minted}
\end{block}
\begin{verbatim}
PresentationExamples.hs:67:33: error:
    • Couldn't match type ‘SlowClock’ with ‘FastClock’
\end{verbatim}
\end{frame}

\section{Quick introduction to Rhine}

\subsection{Synchronous arrowized FRP}

\begin{frame}[fragile]
\begin{block}<+->{\texttt{Dunai} (Iván Pérez, Henrik Nilsson, MB)}
\begin{minted}{haskell}
data MSF m a b = MSF (a -> m (b, MSF m a b))
\end{minted}
\end{block}
\begin{block}<+->{}
\begin{minted}{haskell}
-- Control.Arrow
(>>>) :: MSF m a b -> MSF m b c -> MSF m a c
(***) :: MSF m a b -> MSF m c d -> MSF m (a,c) (b,d)
arr :: (a -> b) -> MSF m a b
\end{minted}
\end{block}
\begin{block}<+->{}
\begin{minted}{haskell}
-- only dunai
arrM :: (a -> m b) -> MSF m a b
\end{minted}
\end{block}
\begin{itemize}[<+->]
	\item \mintinline{haskell}{MSF (Reader Double)} is a replacement for \texttt{Yampa}.
	\item Other monads allow for concise FRP paradigms:
		\begin{itemize}[<+->]
			\item \mintinline{haskell}{State}, \mintinline{haskell}{Reader} and \mintinline{haskell}{Writer} give global state variables.
			\item \mintinline{haskell}{List} gives branching computations.
			\item \mintinline{haskell}{Either} gives control flow!
		\end{itemize}
	\item Support for (entering and leaving) monad transformers.
\end{itemize}
\end{frame}

\begin{frame}[fragile]
\begin{block}<+->{Arrow syntax}
\begin{minted}{haskell}
{-# LANGUAGE Arrows #-}

verboseSum :: MSF IO Int Int
verboseSum = proc n -> do
  s <- sumS       -< n
  _ <- arrM print -< "The sum is now " ++ show s
  returnA         -< s
\end{minted}
\end{block}
\end{frame}

\subsection{Clocks}

\begin{frame}[fragile]
	% TODO Copy from abstract
	\begin{description}[<+->]
		\item[Clock type] All relevant properties of the clock, such as ...
		\begin{itemize}[<+->]
			\item When, and how often, the clock should tick
			\item Which monad the clock takes side effects in
			\item What additional data (besides a time stamp) the clock outputs
		\end{itemize}
		\item[Clock value] All information needed to run the clock
		\begin{itemize}[<+->]
			\item E.g. physical device address, event socket
			\item Implementation choice
		\end{itemize}
		\item[Running clock] Side-effectful stream of \emph{time stamps}, tagged with additional info about the \emph{tick}.
	\end{description}
\end{frame}


\begin{frame}[fragile]
\begin{block}<+->{\texttt{Rhine}}
\begin{minted}{haskell}
-- simplified here
class Clock m cl where
  type TimeDomainOf cl -- time stamp
  type Tag cl -- additional information about tick
  startClock :: cl -> MSF m () (TimeInfo cl, Tag cl)
\end{minted}
\end{block}

\begin{block}<+->{}
\begin{minted}{haskell}
data TimeInfo cl = {...}
  -- absolute and relative time, tag
\end{minted}
\end{block}
\note<.>{While a clock only needs to output a time stamp,
\texttt{Rhine} calculates absolute and relative (to the start of the program and to the last tick) time.}
\end{frame}


\begin{frame}[fragile]
\visible<+->{A clock produces side effects to...}
\begin{itemize}
	\item[...] wait between ticks,
	\item[...] measure the current time,
	\item[...] produce additional data (e.g. events).
\end{itemize}

\begin{block}<+->{Examples of clocks}
\begin{itemize}
	\item Fixed sample rate (e.g. \mintinline{haskell}{Millisecond n})
	\item Events (e.g. \mintinline{haskell}{Stdin})
\end{itemize}
\end{block}
\end{frame}


\begin{frame}[fragile]
\begin{block}<+->{}
\begin{minted}{haskell}
type SyncSF m cl a b = MSF (ReaderT (TimeInfo cl)) a b
\end{minted}
\end{block}
\begin{block}<+->{Lifting \texttt{dunai} concepts}
\begin{minted}{haskell}
arrMSync :: (a -> m b) -> SyncSF m cl a b
\end{minted}
\end{block}
\begin{block}<+->{Time information}
\begin{minted}{haskell}
timeInfo :: SyncSF m cl () (TimeInfo cl)
\end{minted}
\end{block}
\begin{block}<+->{Basic signal processing}
\begin{minted}{haskell}
integral :: VectorSpace v => SyncSF m cl v v
\end{minted}
\end{block}
\end{frame}

\subsection{Exceptions and control flow}

\begin{frame}[fragile]
\begin{block}<+->{\mintinline{haskell}{ExceptT}...}
\begin{minted}{haskell}
--base, transformers
data Either e a = Left e | Right a
newtype ExceptT e m a = ExceptT (m (Either e a))
\end{minted}
\end{block}

\begin{block}<+->{...control flow! (Thanks to Paolo Capriotti)}
\begin{minted}{haskell}
-- dunai, rhine (simplified)
newtype SyncExcept m cl a b e
  = SyncExcept (SyncSF (ExceptT e m) cl a b)
\end{minted}
\end{block}

\begin{block}<+->{}
\begin{minted}{haskell}
instance Monad m => Monad (SyncExcept m cl a b)

throwOn' :: SyncSF (ExceptT e m) cl (Bool, e) ()
try :: SyncSF (ExceptT e m) cl a b
    -> SyncExcept m cl a b e
safely :: SyncExcept m cl a b Empty -> SyncSF m cl a b
safe :: SyncSF m cl a b -> SyncExcept m cl a b e
\end{minted}
\end{block}
\end{frame}

\begin{frame}[fragile]
\begin{block}<+->{Hello World!}
\begin{minted}{haskell}
type SumClock = Millisecond 100

fillUp :: SyncSF (ExceptT Double m) SumClock Double ()
fillUp = proc x -> do
  s <- integral -< x
  _ <- throwOn' -< (s > 5, s)
  returnA       -< ()

helloWorld :: SyncExcept IO SumClock () () Empty
helloWorld = do
  try $ arr (const 1) >>> fillUp
  once_ $ putStrLn "Hello World!"
  helloWorld

main = flow $ safely helloWorld @@ waitClock
\end{minted}
\end{block}
\end{frame}

\begin{frame}[fragile]
\begin{block}{Clock safety}
\begin{minted}{haskell}
fastSignal    :: SyncSF m FastClock () a

slowProcessor :: SyncSF m SlowClock    a b

clockTypeError = fastSignal >>> slowProcessor
\end{minted}
\begin{verbatim}
PresentationExamples.hs:67:33: error:
    • Couldn't match type ‘SlowClock’ with ‘FastClock’
\end{verbatim}
\end{block}
\end{frame}

\subsection{Asynchronous FRP}

\begin{frame}[fragile]
\begin{block}<+->{}
\begin{minted}{haskell}
data Schedule m cl1 cl2
\end{minted}
\end{block}
	\begin{block}<+->{Binary schedules}
		Execute two different clocks simultaneously.
	\end{block}
	\begin{itemize}[<+->]
		\item Can be clock-polymorphic or specific to certain clocks.
		\item (No implementation details here.)
		\item Some examples:
		\begin{itemize}[<+->]
			\item \mintinline{haskell}{concurrently :: Schedule IO cl1 cl2}
			\item \mintinline{haskell}{schedule :: Schedule (ScheduleT m) cl1 cl2}
		\end{itemize}
	\end{itemize}
\end{frame}


\begin{frame}[fragile]
\begin{block}<+->{}
\begin{minted}{haskell}
data ResamplingBuffer m cla clb a b = ResamplingBuffer
  { put :: TimeInfo cla -> a
        -> m (   ResamplingBuffer m cla clb a b)
  , get :: TimeInfo clb
        -> m (b, ResamplingBuffer m cla clb a b)
  }
\end{minted}
\end{block}
\begin{block}<+->{Resampling buffers}
Buffer data at the boundary between two asynchronous systems.
\end{block}
	\begin{itemize}[<+->]
		\item Can be clock-polymorphic or specific to certain clocks.
		\item Some examples
		\begin{itemize}[<+->]
			\item \mintinline{haskell}{collect :: ResamplingBuffer m cl1 cl2 a [a]}
			\item \mintinline{haskell}{fifo :: ResamplingBuffer m cl1 cl2 a (Maybe a)}
			\item \mintinline{haskell}{keepLast :: a -> ResamplingBuffer m cl1 cl2 a a}
			\item Linear interpolation, combinators to build your own\dots
		\end{itemize}
	\end{itemize}
\end{frame}

\begin{frame}[fragile]
\begin{block}<+->{Asynchronous signal functions}
\begin{minted}{haskell}
data SF m cl a b
type family Leftmost cl
type family Rightmost cl
\end{minted}
\end{block}
\begin{block}<+->{A clocked reactive program}
\begin{minted}{haskell}
data Rhine m cl a b
\end{minted}
(...basically an \mintinline{haskell}{SF} and a matching clock!)
\end{block}
\begin{block}<+->{Execution (reactimation)}
\begin{minted}{haskell}
flow :: Rhine m cl () () -> m ()
\end{minted}
\end{block}
\end{frame}

\begin{frame}[fragile]
\begin{block}<+->{Synchronous subsystems}
\begin{minted}{haskell}
cl :: MyClock
syncsf :: SyncSF m MyClock A B
rhineSync :: Rhine m MyClock A B
rhineSync = syncsf @@ cl
\end{minted}
\end{block}
\end{frame}

\begin{frame}[fragile]
\begin{block}<+->{Parallel composition}
\begin{minted}{haskell}
clL :: MyClockL
clR :: MyClockR
syncsfL :: SyncSF m MyClockL C D
syncsfR :: SyncSF m MyClockR C D
schedPar :: Schedule m MyClockL MyClockR
rhinePar = syncsfL @@ clL **@ schedPar @** syncsfR @@ clR
\end{minted}
\end{block}
\begin{block}<+->{Sequential composition}
\begin{minted}{haskell}
buf :: ResamplingBuffer m MyClock (Leftmost (..)) B C
schedSeq :: Schedule m ...
rhineSeq = rhineSync >-- buf -@- schedSeq --> rhineP
\end{minted}
\end{block}
\end{frame}

\begin{comment}
	Introduction what Rhine can do
	Basic vocab
		Arrowized FRP
		Synchronous subsystems
		Clocks
		Schedules
		Resampling buffers
		Events and behaviours
	Comparison (differences, +, -) to
		Yampa
		dunai, auto, varying
		classical FRP
		Pipes, conduit
	Further plan
		Mission statement what we want to implement
			Ideas
				FRP zoo
					Extra features?
				Tea time
					First without, then with GUI
		Given enough time, implement audience suggestions
\end{comment}

\section{Let's hack!}

\begin{frame}
\begin{block}<+->{The plan: A tea app}
	\begin{itemize}
		\item Run several tea timers in parallel
		\item Reactively read tea requests from the console
	\end{itemize}
\end{block}
\visible<+->{\textbf{Any questions before we start hacking?}}
\end{frame}

\begin{frame}
\textbf{Have fun!}
\end{frame}

\section{After the tutorial}

\begin{frame}
	\begin{block}<+->{What else you could (easily) do with \texttt{Dunai} and \texttt{Rhine}}
		\begin{itemize}
			\item Simple arcade games
			\item Reactive console apps
		\end{itemize}
	\end{block}
	\begin{block}<+->{What should be doable, but I didn't do yet because of lazyness}
		\begin{itemize}
			\item Webservers
			\item Interactive File I/O
			\item GUI programs
		\end{itemize}
	\end{block}
	\begin{block}<+->{What might eventually be feasible}
		\begin{itemize}
			\item Reactive audio synthesis, processing and analysis (performance...)
			\item Reactive web apps (GHCJS...)
			\item Android, embedded systems (recent GHC...)
		\end{itemize}
	\end{block}
\end{frame}

\begin{frame}
	\begin{block}<+->{More information}
		\begin{itemize}
			\item \texttt{Dunai}
			\begin{itemize}
				\item  \href{https://github.com/ivanperez-keera/dunai}{github.com/ivanperez-keera/dunai}
				\item There's a link to the article!
			\end{itemize}
			\item \texttt{Rhine}
			\begin{itemize}
				\item \href{https://github.com/turion/rhine-tutorial/}{github.com/turion/rhine-tutorial/}
				\item Checkout branch \texttt{final} for solutions
				\item Documentation on hackage
				\item Simple examples in \href{https://github.com/turion/rhine/}{github.com/turion/rhine/}
			\end{itemize}
		\end{itemize}
	\end{block}
	\begin{block}<+->{What you can do}
		\begin{itemize}
			\item Use \texttt{Rhine} at the hackathon and win a bar of chocolate!
			\item Create issues on \href{https://github.com/turion/rhine/}{github.com/turion/rhine/} and ask for your most needed clocks, schedules, resampling buffers etc.!
			\item Look at easy to solve issues on \href{https://github.com/ivanperez-keera/dunai}{github.com/ivanperez-keera/dunai}!
		\end{itemize}
	\end{block}
\end{frame}

\end{document}
